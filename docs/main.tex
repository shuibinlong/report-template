\documentclass{misc/elegantpaper}
\usepackage{misc/zh_CN-Adobefonts_external} % Simplified Chinese Support using external fonts
\usepackage{indentfirst}
\usepackage{fancyhdr}

\title{基于大数据挖掘的计算机研究综述}
\author{姓名:龙水彬\ \ 学号:1120161842}
\date{}
\pagestyle{fancy}
\fancyhf{}
\fancyhead[C]{北京理工大学计算机学院《计算机新技术专题》总结报告\ \ —\ \ 基于大数据挖掘的计算机研究综述}
\lfoot{}
\cfoot{\thepage}
\rfoot{}

\begin{document}

\maketitle

\textbf{摘要:}数据是日新月异的,在2000年左右,互联网发展的早期,虽然每天产生着大量的新数据,但是数据量相对而言,还是可以用人力来分析和处理的,并且从某个固定的网站角度切入分析,它所需要处理的数据量就更少。然而随着互联网的飞速发展,每天产生的全新数据越来越多,并且呈指数形态增加,大量的数据必然蕴含着大量有价值的信息,它对社会企业、个人发展都有巨大的潜在价值。在这样的背景下,随着计算机硬件技术、制作工艺和算法科学理论知识的不断发展,许许多多数据挖掘处理方法随之应运而生。数据挖掘即使用计算机工具从海量的数据中挖掘出有价值的模式和规律,并用这些模式和规律去预测和指导未来的行为。在当今的互联网背景之下,最为常用的数据挖掘算法有频繁模式挖掘、聚类分析、决策树和贝叶斯网络等,而由数据挖掘而展开的分支领域多达五十余种。本文致力于整体条理系统地介绍数据挖掘的同时引入各分支的详细分析,包括原理、方法、学习路线和应用领域等层面,并在本文尾部介绍北京理工大学数据挖掘的具体研究方向,力求将数据挖掘领域系统详尽地展示报告出来。

\textbf{关键词:}计算机、数据挖掘技术、开发、应用、研究

\section{简介}

伴随着互联网的发展,人类科技水平的进步,应用于海量数据的存储与研究的产物“数据挖掘”诞生。虽然这一技术在学科上尚未有一个统一的、精准的定义,但整体上,我们认同数据挖掘(Data Mining)是一个跨学科的计算机科学分支。它是用人工智能、机器学习、统计学和数据的交叉方法,在相对较大型的数据集中发现模式的计算过程。它的总体目标是从一个数据集中提取信息,并将其转换成可理解的结构,以进一步地使用。除了原始的分析步骤,数据挖掘还涉及到数据库和数据管理方面,数据预处理、模型与推断方面的考量、兴趣度的度量、复杂度的考虑,以及发现可视化、结构及在线更新等后处理。数据挖掘它是“数据库知识发现”(KDD)的分析步骤,从本质上说,它属于机器学习的范畴。

数据挖掘的实际工作,是对大规模数据的自动或半自动分析,以提取过去未知的、有价值的潜在信息。行为上,可以实现数据分组、异常记录、数据间关系挖掘等等,它常涉及的六大任务是“异常检测”、“关联规则学习”、“知识聚类”、“数据分类”、“数据回归”、“数据汇总”,每一种任务都涉及着相应的数据库技术。而对于数据的进一步分析与处理,可以应用在机器学习和预测分析中,建立相应的神经网络而达到更多层面的效果。而如果国内互联网公司的数据挖掘人员工作领域大致可以分为三类:

\begin{itemize}[noitemsep]
      \item \textbf{数据分析师:}从事金融、电信、电商、咨询等行业的业务咨询,商务智能等工作,主要负责分析数据并撰写分析报告。
      \item \textbf{数据挖掘工程师:}在多媒体、电商、搜索、社交等大数据相关行业里继续做机器学习、算法实现和数据分析,这部分人员也最贴近数据挖掘的本质工作。
      \item \textbf{科学研究方向:}在高校、科研单位、社会的企业研究院等科研机构研究所进行研究,主要针对新的算法、效率改进等应用方向。
\end{itemize}

在数据挖掘大方向下,计算机科学家们如今钻研出许多成熟的理论体系,每个理论体系都有着相应的应用领域与应用实例,更是现如今崇尚计算机科学的一批批大学、研究生、博士生学者们探索的大热门课题之一,接下来本文也将重点介绍计算机科学数据挖掘领域颇有成就的一些方向和其相应的研究成果、对应的标准学习方向路线。


\section{算法}

当讨论到数据挖掘的实现算法,从分类上看,有“监督式学习”、“非监督式学习”、“半监督式学习”和“增强学习这几种方法,下面简要介绍这几类算法分支。

\subsection{监督式学习算法}

监督式学习(Supervised learning)是机器学习的一种常见方法,它由训练资料中学到或建立一个新的模式函数,并由该模式函数做处理接口用以预测新的实例。

完整的监督式学习模式,是通过观测训练资料的输入和预期输出,最终归纳出更完整更“智能”的函数。因此,这个算法模式最重要的部分便是训练资料,它由输入物件向量和预期输出组成。通过改变不同类型的训练资料,可以实现监督室学习的不同目的,而通过改变不同的训练方法(分类器),更可以实现监督室学习的不同输出,常用的分类器包括人工神经网络、支持向量机、最近邻居法、高斯混合模型、朴素贝叶斯方法、决策树和径向基函数分类等。

监督式学习在大数据挖掘处理中,常常用来做数据分类、数据估计和预测。

\subsection{非监督式学习算法}

相对于监督式学习而言,它并不需要人力来输入标签(即不需要认为的提供预期输出),而且通过少量范例和强化学习的模式,进行大量可靠的推测,现如今也广泛应用于数据聚类。在人工神经网络中,生成对抗网络(GAN)、自组织映射(SOM)和适应性共振理论(ART)则是最常用的非监督式学习。

非监督式学习常用的方法有很多种,包括以“K-平均算法”、“混合模型”、“阶层式分群”为主的分群法,也包括以“自编码”、“深度置信网络”、“赫布学习”等算法为主的人工神经网络算法,还包括以“最大期望算法”、“矩估计”、“盲信号分离技术”算法为主的学习潜在变数模型算法。

\subsection{增强学习领域}

增强学习又叫强化学习,是隶属于机器学习的一个领域,它强调如何基于环境而行动以取得最大化的预期利益,其灵感来自于心理学的行为主义理论(即有机体如何在环境给与的奖励或惩罚的刺激下,逐步形成对刺激的预期,产生能获得最大利益的习惯性行为)。强化学习的基本学习模型包括:
\begin{itemize}[noitemsep]
      \item 环境状态的集合 $S$
      \item 动作的集合 $A$
      \item 在状态之间的转换规则
      \item 在规定转换后“即时奖励”的规则
      \item 描述主体能够观察到什么的规则
\end{itemize}

而这里的规则,常常是随机生成的,主体通常可以观察即时奖励和最后一次转换。强化学习的主体与环境基于离散的时间步长相作用。在每一个时间 $t$,主体接收到一个观测 $o_t$,通常其中包含奖励 $r_t$。然后,它从允许的集合中选择一个动作 $a_t$,然后送出到环境中去。环境则变化到一个新的状态 $s_{t+1}$,然后决定了和这个变化 $(s_t,a_t,s_{t+1})$ 相关联的奖励 $r_{t+1}$。强化学习主体的目标,是得到尽可能多的奖励。主体选择的动作是其历史的函数,它也可以选择随机的动作。

在许多的强化学习模型中,主体被假设为可以观察现有的环境状态,这种情况称为“完全可观测”(full observability),反之则称为“部分可观测”(partial observability)。有时,主体被允许的动作是有限的(例如,你使用的钱不能多于你所拥有的)。

说到强化学习的搜索机制,就不得不说强化学习的聪明机制,包括随机采样法和$\epsilon$贪心法,前者是通过在解空间中随机采样,在小规模马尔科夫过程中很适用,但是到了大规模问题上,后者$\epsilon$贪心法中$\epsilon$是一个可调节参数,在每一步决策中,以较大的概率($1-\epsilon$)去选择现在最好的动作,而如果没有最优动作则再进行随机选择,显然在大规模问题上,这个算法更容易找到最优解。

强化学习广泛应用于各个学科领域,包括博弈论、控制论、运筹学、信息论、仿真优化、多主体系统学习、群智能学习、统计学以及遗传算法等。而智能控制方面,机器人控制、电梯调度、电信通讯、双陆棋和西洋跳棋方面强化学习更是能够做到精确计算。


\section{分支}

当讨论到数据挖掘的分支情况时候,从不同的侧面我们可以分成不同的分支类。刚刚从算法层面,已经简要剖析了数据挖掘下的一些常用算法,那些算法不仅仅应用在数据分析挖掘领域中,更是在机器学习的大领域中广受青睐。

而从学科应用领域细分数据挖掘这个大方向,也有许许多多的分支,我们大致可以分为科学领域研究和商业领域研究这两种,而科学领域上,数据挖掘主要应用于各个学科的数据分析,包括生物学的行为信息学和基因仓库数据分析学、医学数据分析学、基础数理化学科的解析探索性数据分析,而商业研究领域主要模式是以互联网为中心的互联网大数据挖掘研究,包括网络挖掘、数据分析、商业智能等等。

但是各个领域分支之间,并不是相互独立的,有很大一部分内容是相通的算法,相通的理论知识,本节分支介绍主要针对一些当前热点研究领域进行逐一介绍。

\subsection{基于大数据的数据分析领域}

随着互联网的发展,海量数据已经成为了一种伴随而来的特征,大数据处理的领域应用于方方面面,包括大科学、感测设备网络、天文学、大气学、交通运输、基因组学、生物学、大社会数据分析、互联网文件处理、制作互联网搜索引擎索引、通信记录明细、军事侦查、金融大数据,医疗大数据,社交网络、通勤时间预测、医疗记录、照片图像和视频封存、大规模的电子商务等。

而其社会应用之广泛,不论是科研领域还是商业领域,都在持续关注着这方面研究。随之带来的也是相关的安全隐患,各个国家和地区也针对大数据正出台着一系列的政策和法案以填充数据隐私、数据权限等这方面的空白。

在科学研究领域,生物工程信息学研究更是可以利用数学相应的计算和统计方法进行大量的研究,围绕基因和蛋白质建立神经网络和海量数据,使预测生物状态和验证生物理论假说成为可能。而高超的计算能力和可靠的推测模型更让基础数理化学科的演算和推理计算能力更进一步,即使在现如今尚未解决的史上NP难题,将来也可能通过智能的大数据挖掘分析算法进行近似求解,满足人类需求。

\subsection{智能数据挖掘分析领域}

随着计算机硬件技术的发展,硬件设备(尤其指CPU和GPU方面)计算能力的提升,使得过去科学家们限于计算能力有限而搁置的人工智能领域重新火热起来。基于深度学习领域下的学习技术越发地成熟,结合数据挖掘技术衍生的人工智能产品近年也层出不穷。

涉及数据挖掘的智能领域,主要做的工作就是将海量的数据中提取知识,因此需要利用数据仓库、在线分析处理(OLAP)工具和数据挖掘技术,它是基于ETL、数据仓库、OLAP、数据挖掘、数据展现等技术的综合应用。而智能商业化体现在从许多来自不同的企业运作系统的数据中提取出有用的数据并进行清理,以保证数据的正确性,然后经过抽取(Extraction)、转换(Transformation)和装载(Load),即ETL过程,合并到一个企业级的数据仓库里,从而得到企业数据的一个全局视图,在此基础上利用合适的查询和分析工具、数据挖掘工具、OLAP工具等对其进行分析和处理(这时信息变为辅助决策的知识),最后将知识呈现给管理者,为管理者的决策过程提供支持。具体应用的领域包括了商业客户分析、科学药物发现等等。

集科学研究和商业应用为一体的最典型的公司便是举世闻名的互联网公司Google,他旗下的Google产品让全球人受益,大家最熟悉谷歌翻译产品“Translation Google”,即使是一句简单的短语翻译工作,也是基于深度学习在互联网海量的权威语言词典学习后的函数输出的结果。中国在这方面领域也有着出色的世界前沿技术,以百度、字节跳动等互联网大型公司为首的智能化引擎也已经植入我们生活的方方面面。2018年中旬,从事智能车领域研究的Pony.ai首席主席更是宣布智能车成功上路并通过了全国最严格的智能车驾考,这也宣布着在智能汽车领域我们迈出了一大步。我们有理由相信未来人类将在这个领域取得更大更瞩目的突破!

\section{学习路线}

当学习一门计算机技术时,首先需要向行业靠拢,了解行业发展详情,国内技术背景。如果时间充裕,再在整体把握技术脉络基础上从重点细节知识着手学习。数据挖掘的主干脉络便是理论基础,因此学好数学和计算机科学的基础知识是成功入门的前置技能要求。再者便是深入学习机器学习某个分支领域,从相关前沿的论文中挖掘出有价值的方向。当然社会职业必然会要考虑其商业价值,而商业价值的来源便是客户的需求,因此更需要时刻把握市场客户的需求,积累行业经验。

理论知识模块的学习首先需要掌握的就是数学基础,数学基础方面包括代数知识和统计学知识都是必修知识。因此“线性代数”、“统计学”、“最优化”属于这方面核心。因为计算机能处理的代数运算大都是以矩阵的形式,因此良好的线性代数基础是必备技能,这方面知名的教材便是由美国著名教授Horn编写的《矩阵分析》。理论知识还有的重要的一环便是统计学中的概率知识,这也是机器学习里理论推导的核心和基础,统计学学习教材国内外也许许多多,最经典的便是美国教授门登霍尔的《统计学》,现如今很多高校也专门开设了统计学的一门“概率论与数理统计”课程作为工科生的必修课程之一。当涉及到算法层面的优化,很多机器学习算法最终都可以变成“条件-目标函数”的形式,“最优化”做的一件事情就是对目标函数进行参数求解,算法层面的学习应该循序渐进,国内高校的培养模式是从基础语言切入(往往是C语言或者Python语言),接着深入离散数学等代数内容,接着教授数据结构算法分析等课程,最后引入最优化相关课程进行进阶培养。因此从事机器学习领域下的数据挖掘方向,首先需要相当扎实的计算机数学理论基础,才能做到深入时候能知其所以然。

当理论知识扎实后就可以考虑开始着手数据挖掘专业方面的学习,此时《数据挖掘导论》就是很不错的选择,它以实际应用为主,系统地介绍了数据挖掘的各种分支和相应的应用,深入浅出。同时掌握机器学习相关的经典知识也是必不可少的,清华大学周志华老师的《机器学习》(西瓜书)、李航老师的《统计学习方法》就是非常好的入门书籍,而且二者虽然涉及的方向点大致类似,但是侧重的细节各不相同,对照着学习能够对这方面理论更加地熟练。

当然光掌握理论知识和专业基础是远远不够的,计算机专业很重要的部分就是实践。以上的学习内容依然只是属于大学本科入门的初级阶段,深入学习和应用才能化知识为自己的力量。而且机器学习数据挖掘在程序设计方面的应用已经相当成熟,市面上有许许多度改良的库函数和脚本,因此熟练掌握和应用用于数据挖掘的编程语言,在实战中磨练自己的技能才能说是真正地踏入了领域。

\section{本校现状}

身为数据挖掘领域的爱好者,北京理工大学计算机科学与技术的本科生,笔者于大学二年级阶段便开始关心本校机器学习数据挖掘方向领域的实验室力量。

李荣华老师是我校的图数据挖掘实验室的博士生导师,主要从事基于图数据挖掘、社交网络分析与挖掘、基于图的机器学习和图数据库、图计算系统等方面的研究。图数据挖掘实验室现有3名博士生,5名硕士研究生,主要的研究成果包括研究了数据库领域的一个经典NP难问题--群组斯坦纳树问题(Group Steiner Tree problem),发现和证明了一个树结构分解定理,基于此设计和开发了一个“分治+A星”的算法,该算法较之前最快的精确算法要快2个以上数量级,内存消耗要少近2个数量级;算法甚至较之前最快的近似算法都要快1个以上数量级,同时在神经网络上,优化了社交网络中的社区挖掘和搜索问题,提出了影响力社区、天际线社区、持久性社区、符号社区等一系列全新的社区模型和图论概念,并设计了一系列高效快速的社区挖掘和搜索算法。

宋丹丹老师是我校博士生导师,主要科研方向是深度学习、信息检索、数据挖掘和生物信息方向,其个人在国内外期刊及会议上发表的论文专利有三十余项目,主要研究成果包括基于生物基因学的大数据分析以智能识别疾病基因和基于海量互联网数据深度学习模型下的情感分析等项目。

辛欣老师毕业于清华大学本科生和香港中文大学博士后,现实验室团队从事基于深度学习技术的自然语言处理(包括知识图谱、对话系统、新闻检索及推荐),和数据挖掘(主要方向是智能交通和时空序列预测),近年研究项目包括融合知识图谱的文本个性化推荐机制、融合异构信息的低秩分解推荐模型研究和跨语言微博主题挖掘在线模型研究等。

除了上面的导师之外,北京理工大学计算机学院和其他学院也都还有着许许多多从事数据挖掘、机器学习方向的优秀导师。随着互联网的持续扩张,数据挖掘正渗透着各门学科各个领域,也越来越成为高校的计算机科学重点研究方向。

\section{个人想法}

作为一个从大一就开始参加ACM/ICPC程序设计竞赛的CS大学生来说,我对算法充满着感情,深刻地认识到算法在程序设计的地位。但是有别于程序设计竞赛,在机器学习领域,抛开那些优美的算法,奇妙的数学公式,背后还有着很深刻的社会意义和研究价值。如今大三的我,除了继续奋斗我的ICPC竞赛生涯外,我也选择加入了李荣华教授的图论大数据挖掘实验室,从事图数据挖掘算法优化的研究工作,在学习博士生学长经验的同时,积攒我在数据挖掘领域的经验。我也愿意在不久的将来,研究生领域继续从事机器学习数据挖掘相关领域的研究,为相关研究工作贡献自己的一份力量!

\section{参考文献}

\noindent [1] 黄永毅.计算机数据挖掘技术的开发及其应用.

\noindent [2] 张佳.计算机数据挖掘技术及其应用探析.

\noindent [3] 葛俊言.数据挖掘技术的应用研究.

\noindent [4] ACM SIGKDD. Data Mining Curriculum.

\noindent [5] \href{https://en.wikipedia.org/wiki/Machine_learning}{Wikipedia}. Machine learning.

\noindent [6] \href{https://ronghuali.github.io/}{Li Ronghua Blog}.北京理工大学数据挖掘实验室介绍.

\noindent [7] \href{http://cs.bit.edu.cn/szdw/jsml/js/sdd_20170821091432439000/index.htm}{BITCS Introduction}.宋丹丹教授简介.

\noindent [8] \href{http://cs.bit.edu.cn/szdw/jsml/fjs/xx/index.htm}{BITCS Introduction}.辛欣教授简介.

\end{document}
